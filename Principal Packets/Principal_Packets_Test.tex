\documentclass[sfsidenotes, justified]{tufte-handout}\usepackage{graphicx, color}
%% maxwidth is the original width if it is less than linewidth
%% otherwise use linewidth (to make sure the graphics do not exceed the margin)
\makeatletter
\def\maxwidth{ %
  \ifdim\Gin@nat@width>\linewidth
    \linewidth
  \else
    \Gin@nat@width
  \fi
}
\makeatother

\IfFileExists{upquote.sty}{\usepackage{upquote}}{}
\definecolor{fgcolor}{rgb}{0.2, 0.2, 0.2}
\newcommand{\hlnumber}[1]{\textcolor[rgb]{0,0,0}{#1}}%
\newcommand{\hlfunctioncall}[1]{\textcolor[rgb]{0.501960784313725,0,0.329411764705882}{\textbf{#1}}}%
\newcommand{\hlstring}[1]{\textcolor[rgb]{0.6,0.6,1}{#1}}%
\newcommand{\hlkeyword}[1]{\textcolor[rgb]{0,0,0}{\textbf{#1}}}%
\newcommand{\hlargument}[1]{\textcolor[rgb]{0.690196078431373,0.250980392156863,0.0196078431372549}{#1}}%
\newcommand{\hlcomment}[1]{\textcolor[rgb]{0.180392156862745,0.6,0.341176470588235}{#1}}%
\newcommand{\hlroxygencomment}[1]{\textcolor[rgb]{0.43921568627451,0.47843137254902,0.701960784313725}{#1}}%
\newcommand{\hlformalargs}[1]{\textcolor[rgb]{0.690196078431373,0.250980392156863,0.0196078431372549}{#1}}%
\newcommand{\hleqformalargs}[1]{\textcolor[rgb]{0.690196078431373,0.250980392156863,0.0196078431372549}{#1}}%
\newcommand{\hlassignement}[1]{\textcolor[rgb]{0,0,0}{\textbf{#1}}}%
\newcommand{\hlpackage}[1]{\textcolor[rgb]{0.588235294117647,0.709803921568627,0.145098039215686}{#1}}%
\newcommand{\hlslot}[1]{\textit{#1}}%
\newcommand{\hlsymbol}[1]{\textcolor[rgb]{0,0,0}{#1}}%
\newcommand{\hlprompt}[1]{\textcolor[rgb]{0.2,0.2,0.2}{#1}}%

\usepackage{framed}
\makeatletter
\newenvironment{kframe}{%
 \def\at@end@of@kframe{}%
 \ifinner\ifhmode%
  \def\at@end@of@kframe{\end{minipage}}%
  \begin{minipage}{\columnwidth}%
 \fi\fi%
 \def\FrameCommand##1{\hskip\@totalleftmargin \hskip-\fboxsep
 \colorbox{shadecolor}{##1}\hskip-\fboxsep
     % There is no \\@totalrightmargin, so:
     \hskip-\linewidth \hskip-\@totalleftmargin \hskip\columnwidth}%
 \MakeFramed {\advance\hsize-\width
   \@totalleftmargin\z@ \linewidth\hsize
   \@setminipage}}%
 {\par\unskip\endMakeFramed%
 \at@end@of@kframe}
\makeatother

\definecolor{shadecolor}{rgb}{.97, .97, .97}
\definecolor{messagecolor}{rgb}{0, 0, 0}
\definecolor{warningcolor}{rgb}{1, 0, 1}
\definecolor{errorcolor}{rgb}{1, 0, 0}
\newenvironment{knitrout}{}{} % an empty environment to be redefined in TeX

\usepackage{alltt}
\usepackage{url}
\usepackage[english]{babel}
\usepackage{blindtext}


\title{Principal Packets Test}
\author{Christopher J. Haid}


\makeatother


\begin{document}




\maketitle
\begin{abstract}
This document is a beta version of KIPP:Chicago's \textbf{Principal Packet}.  This proposed principal packet has two aims.  First, and most importantly, to present our school leaders with actionable data and analysis to help our students achieve at the highest level.  Secondly, is to provide a proof-of-concept that will move regional reporting \emph{away from} PowerPoint style decks and towards more thoughtful, insightful, and informative reports.  To this end, this report needs to include useful, clear data visualization that are explained and contextualized with incisive analytical prose, rather than the near meaningless, overly-simplified prolix of the dreaded bullet point.\sidenote{For more on the pitfalls of PowerPoints ``cognitive style'' please read Edward Tufte's excellent \emph{The Cognitive Style of PowerPoint} (\url{http://bit.ly/SuaNBh}).}
\end{abstract}





















  











\section{Are we serving the children who need us?}
This section should have a side table with region wide demographics from :
  Ethnicity (by school?)
  Gender
  FRL
  SPED
  ELL
  E





We currently enroll **X** students of whom **Y** are female and **Z** are male. 
\begin{margintable}

% latex table generated in R 2.15.1 by xtable 1.7-0 package
% Mon Nov 19 22:04:00 2012
{\small
\begin{tabular}{lllr}
  \hline
Grade & Girls & Boys & Total \\ 
  \hline
K & 49 (47\%) & 49 (53\%) & 105 \\ 
  1st & 49 (46\%) & 49 (54\%) & 106 \\ 
  2nd & 51 (48\%) & 51 (52\%) & 106 \\ 
  5th & 97 (57\%) & 97 (43\%) & 173 \\ 
  6th & 49 (55\%) & 49 (45\%) &  89 \\ 
  7th & 43 (49\%) & 43 (51\%) &  88 \\ 
  8th & 45 (55\%) & 45 (45\%) &  82 \\ 
  Total & 383 (51\%) & 383 (49\%) & 749 \\ 
   \hline
\end{tabular}
}



\caption{KIPP Chicago Enrollment by Gender}
\end{margintable}
  
Racially we our students are Y\% African American and Z\% Latino.  X\% qualify for free or reduced lunch and S\% have an accomodation.

\begin{margintable}[7\baselineskip]

% latex table generated in R 2.15.1 by xtable 1.7-0 package
% Mon Nov 19 22:04:00 2012
{\small
\begin{tabular}{lllr}
  \hline
Grade & Black & Latino & Total \\ 
  \hline
K & 99 (94\%) & 6 (6\%) & 105 \\ 
  1st & 100 (94\%) & 6 (6\%) & 106 \\ 
  2nd & 98 (92\%) & 8 (8\%) & 106 \\ 
  5th & 170 (98\%) & 3 (2\%) & 173 \\ 
  6th & 89 (100\%) & 0 (0\%) &  89 \\ 
  7th & 87 (99\%) & 1 (1\%) &  88 \\ 
  8th & 81 (99\%) & 1 (1\%) &  82 \\ 
  Total & 724 (97\%) & 25 (3\%) & 749 \\ 
   \hline
\end{tabular}
}



\caption{KIPP Chicago Enrollment by Ethnicity}
\end{margintable}



Perhaps more pertinent the question denoting this section is to ask at what level of academic achievemnt are our  students newly entering our schools and how does that compare to national levels of achievement?





\begin{figure}
\begin{knitrout}
\definecolor{shadecolor}{rgb}{0.969, 0.969, 0.969}\color{fgcolor}

{\centering \includegraphics[width=\linewidth]{figure/MAP_KAMS_KCCP_Histo_Math} 

}


\end{knitrout}

\caption{KAMS and KCCP 5th Grade Distribution of RIT Scores versus the National Distribution of RIT Scores, Fall 2012}
\end{figure}


\begin{figure}
\begin{knitrout}
\definecolor{shadecolor}{rgb}{0.969, 0.969, 0.969}\color{fgcolor}

{\centering \includegraphics[width=\linewidth]{figure/MAP_KAPS_Histo_Math} 

}


\end{knitrout}

\caption{KAPS Kindergarten Distribution of RIT Scores versus the National Distribution of RIT Scores, Fall 2012}
\end{figure}


  
  
\section{Are our students staying with us?}
This section should have a simply graph showing the proportions for kids leaving.  We should probably also look at numbers leaving and reasons by grade (Are we losing kids at a predictable point?).  A table should be included in a side bar showing reasons, counts, and percentages.

Last year 
\begin{knitrout}
\definecolor{shadecolor}{rgb}{0.969, 0.969, 0.969}\color{fgcolor}\begin{kframe}
\begin{verbatim}
##    School_ID Exit_Reason count      pct
## 1       7810       2e+00     9 0.016760
## 2       7810       3e+00     7 0.013035
## 3       7810       5e+00     1 0.001862
## 4       7810       6e+00     1 0.001862
## 5       7810       7e+00     1 0.001862
## 6       7810       9e+00     3 0.005587
## 7       7810       1e+01     6 0.011173
## 8       7810       1e+06    77 0.143389
## 9       7810          NA   231 0.430168
## 10     78102       2e+00     6 0.011173
## 11     78102       3e+00    11 0.020484
## 12     78102       6e+00     1 0.001862
## 13     78102       7e+00     2 0.003724
## 14     78102       9e+00     1 0.001862
## 15     78102       1e+01     4 0.007449
## 16     78102          NA   176 0.327747
\end{verbatim}
\end{kframe}
\end{knitrout}





Historical mobility and attrition data

Enrollment and attendence

\section{Are our students progressing and achieving academically?}

ISAT last year
MAP Results from last year
MAP Results this year

Look at percentile/quartile movement fall to fall?  

Results by sped?

\section{Are we supporting kids to and through collegte?}

Graph of Selective versus no selective.

\section{Do we attract and retain talented educators?}

Teacher attrition rates
Graph/table of exit reasons

Q12 and HSR results 

\section{Are we building a financially sustainable model?}

Probably budget burn data here

\newthought{This section is a test of graphing,} espeically of the marginal variety. 
\blindtext
\blindtext

\blindtext
\blindtext
\section{Enrollment, Attrition, \& Attendence}
\subsection{Attendence} 









\begin{margintable}

% latex table generated in R 2.15.1 by xtable 1.7-0 package
% Mon Nov 19 22:04:04 2012
{\small
\begin{tabular}{lrrr}
  \hline
Week of & KAMS & KAPS & KCCP \\ 
  \hline
Aug 13 & 0.99 & 0.98 &  \\ 
  Aug 20 & 0.98 & 0.96 &  \\ 
  Aug 27 & 0.96 & 0.97 & 0.99 \\ 
  Sep 3 & 0.94 & 0.96 & 0.98 \\ 
  Sep 10 & 0.95 & 0.95 & 0.91 \\ 
  Sep 17 & 0.97 & 0.96 & 0.93 \\ 
  Sep 24 & 0.96 & 0.96 & 0.97 \\ 
  Oct 1 & 0.97 & 0.98 & 0.99 \\ 
  Oct 8 & 0.94 & 0.96 & 0.97 \\ 
  Oct 15 & 0.94 & 0.96 & 0.98 \\ 
  Oct 22 & 0.96 & 0.97 & 0.96 \\ 
  Oct 29 & 0.96 & 0.96 & 0.98 \\ 
  Nov 5 & 0.96 & 0.97 & 0.96 \\ 
  Nov 12 & 0.93 & 0.94 & 0.95 \\ 
   \hline
\end{tabular}
}



\caption{KIPP Chicago Weekly Attendence Rates}
\end{margintable}



The KIPP Chicago \textbf{year-to-date attendence rate} is **96\%***. The YTD attendance rates for each of the three schools is 96\%, 96\%, and 96\% for KAPS, KAMS, and KCCP, respectively.\sidenote{The school year for for began on August 13, 2012 for KAPS and KAMS and on August 27, 2012 for KCCP.  Consequntly all of the attendance analysis is bassed on data pulled from PowerSchool for the time between August 13 and today (November 6, 2012).}   Table \ref{t:Att_by_week} shows weekly attendence rates for year school as well as YTD attendence rates. 

Daily Enrollement, our consequent daily attendence goal (96\% of Enrollemt), and daily attendence are displayed for each day by week in Figure \ref{f:Enroll_Attend_Daily}.  Clearly the three schools have seen incresing enrollment over the first eight weeks of the school year (through the week of September 24 for KAPS and KAMS and the week of October 15 for KCCP).  

\begin{figure*}[b!]
\begin{knitrout}
\definecolor{shadecolor}{rgb}{0.969, 0.969, 0.969}\color{fgcolor}

{\centering \includegraphics[width=\linewidth]{figure/Plot_Attendance} 

}


\end{knitrout}

\end{figure*}\label{f:Enroll_Attend_Daily}

\subsection{Suspensions}




\begin{margintable}

% latex table generated in R 2.15.1 by xtable 1.7-0 package
% Mon Nov 19 22:04:13 2012
{\small
\begin{tabular}{lrrr}
  \hline
Grade & KAMS & KAPS & Total \\ 
  \hline
2 & 0 & 7 & 7 \\ 
  5 & 2 & 0 & 2 \\ 
  6 & 9 & 0 & 9 \\ 
  7 & 19 & 0 & 19 \\ 
  8 & 4 & 0 & 4 \\ 
  Total & 34 & 7 & 41 \\ 
   \hline
\end{tabular}
}



\caption{KIPP Chicago YTD Suspension Totals by Grade and Sxchool}
\end{margintable}

\begin{margintable}[3\baselineskip]

% latex table generated in R 2.15.1 by xtable 1.7-0 package
% Mon Nov 19 22:04:13 2012
{\small
\begin{tabular}{lrrrrrr}
  \hline
Week of & 2nd & 5th & 6th & 7th & 8th & Total \\ 
  \hline
Aug 13 & 1 & 0 & 0 & 0 & 0 & 1 \\ 
  Aug 27 & 0 & 0 & 0 & 6 & 0 & 6 \\ 
  Sep 10 & 0 & 0 & 4 & 0 & 1 & 5 \\ 
  Sep 17 & 0 & 0 & 3 & 0 & 1 & 4 \\ 
  Sep 24 & 2 & 0 & 2 & 0 & 1 & 5 \\ 
  Oct 1 & 0 & 0 & 0 & 1 & 0 & 1 \\ 
  Oct 8 & 0 & 2 & 0 & 2 & 1 & 5 \\ 
  Oct 15 & 0 & 0 & 0 & 7 & 0 & 7 \\ 
  Oct 22 & 0 & 0 & 0 & 3 & 0 & 3 \\ 
  Oct 29 & 4 & 0 & 0 & 0 & 0 & 4 \\ 
  Total & 7 & 2 & 9 & 19 & 4 & 41 \\ 
   \hline
\end{tabular}
}



\caption{KIPP Chicago Weekly Suspension Totals by Week and Grade}
\end{margintable}

\begin{knitrout}
\definecolor{shadecolor}{rgb}{0.969, 0.969, 0.969}\color{fgcolor}\begin{kframe}
\begin{verbatim}
##    SCHOOLID GRADE_LEVEL                    LASTFIRST
## 1     78102           2               Pride, Brian T
## 2      7810           7       Henderson, Ikea Alicia
## 3      7810           7                 McDavis, Ivy
## 4      7810           7       Henderson, Ikea Alicia
## 5      7810           7                 McDavis, Ivy
## 6      7810           7       Henderson, Ikea Alicia
## 7      7810           7                 McDavis, Ivy
## 8      7810           8             Donahue, Wonshon
## 9      7810           6              Protho, Antonio
## 10     7810           6              Higgins, Pierre
## 11     7810           6              Higgins, Pierre
## 12     7810           6              Protho, Antonio
## 13     7810           6              Higgins, Pierre
## 14     7810           6              Higgins, Pierre
## 15     7810           6              Higgins, Pierre
## 16     7810           8             Donahue, Wonshon
## 17     7810           6                Toney, Davion
## 18    78102           2                 Davis, Brian
## 19     7810           6                Toney, Davion
## 20    78102           2                 Davis, Brian
## 21     7810           8            Campbell, Alchino
## 22     7810           7              Campbell, Aaron
## 23     7810           5             Powell, Sharrond
## 24     7810           7               Currie, Sequan
## 25     7810           7             Cleveland, Ahmad
## 26     7810           8             Donahue, Wonshon
## 27     7810           5       Nolton, Davion Antonio
## 28     7810           7        Nance, Markia Marshae
## 29     7810           7         McCoy, Delois Malika
## 30     7810           7              Campbell, Aaron
## 31     7810           7         Williams Jr., Ronnie
## 32     7810           7         Hollingworth, Symone
## 33     7810           7             Campbell, Andrea
## 34     7810           7               Burns, Terrell
## 35     7810           7 Carldwell, Dashawn Christian
## 36     7810           7 Carldwell, Dashawn Christian
## 37     7810           7 Carldwell, Dashawn Christian
## 38    78102           2               Noble, Vonda'e
## 39    78102           2          Pruitt, Tyriontae D
## 40    78102           2      Williams, Rueben-O'neal
## 41    78102           2               Noble, Vonda'e
\end{verbatim}
\end{kframe}
\end{knitrout}




\end{document}
